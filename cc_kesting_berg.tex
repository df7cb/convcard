%    \begin{macrocode}
\documentclass{article}
\usepackage{miniDBV2,amsthm}
\usepackage[a4paper, landscape, margin=5mm]{geometry}
\usepackage{multicol,tabularx}
\renewcommand{\familydefault}{cmss}
\begin{document}

\newtheorem{convention}{}
\newcommand{\Ref}[1]{$^{\ref{#1}}$}

\hspace{2.6mm}
\beginMinikarteNeu
	{\large Hennes Kesting}{\large Christoph Berg}
%
\Grundsystem{\large Standard American}
\EinSA{\large15-17}{\large15-17}
%\EinSAkleinesSingle
%\EinSATopSingle
%\EinSAFuenferOFregel
%\EinSAFuenferOFselten
\Mindestlaengen{\large2}{\large4}{\large5}{\large5}
\EinTreffBed{2+ \treff, 18er-Regel (in Gefahr 20er)}
\EinTreffAnt{Inverted (auch nach Zwischenreizung)}
	{3./4. Farbe forcing, Assfrage RKC 30/41, platzierte K�nige, 4UF Assfrage bei Fit}
\EinKaroBed{4+ \karo}
\EinKaroAnt{(dito)}
	{}
\EinCoeurBed{5+ \coeur}
\EinCoeurAnt{Splinter (nur direkte Antwort)}
	{}
\EinPikBed{5+ \pik}
\EinPikAnt{(dito)}
	{}
\EinSABed{nat.}
\EinSAAnt{Stayman, Transfers (2\pik{} = \treff, 3\treff{} = \karo)}
	{}
%
\ZweiTreffBed
	{Partieforcing (23+FP oder 9$1\over2$ Spielstiche)}
	{}
\ZweiTreffAnt{2\karo{} Relais, Rest nat. ab 7FP}
\ZweiKaroBed
	{Multi (Weak Two in OF 6-10FP, SA 20-22FP, Semiforcing)}
	{}
\ZweiKaroAnt{2\coeur{} Relais, 2\SA{} fragt (min \coeur/min \pik/max \coeur/max \pik)}
\ZweiCoeurBed
	{Zweif�rber mit \coeur (6-10FP)}
	{}
\ZweiCoeurAnt{2\pik{} Relais, 2\SA{} fragt nach 2. Farbe, 3\coeur{} einl.}
\ZweiPikBed{Zweif�rber \pik{} und UF (6-10FP)}
	{}
\ZweiPikAnt{3\treff{} Relais, 2\SA{} fragt nach 2. Farbe, 3\pik{} einl.}
\ZweiSABed{UF-Zweif�rber (6-10FP)}
	{}
\ZweiSAAnt{3\treff/3\karo zum Spielen}
%
\BesondereZweierUndHoeher{3\SA: Gambling (stehende 7er-UF ohne Nebenwerte)}
	{}
%
\InfoKontraAb{12}
\InfoKontraOF
%\InfoKontraWerte
\FarbGegenEiner{8}{16}
\FarbGegenZweier{11}{17}
\StilDerGegenreizung{konstruktiv}
\Weiterreizung{Farbwechsel forcing}
\EinSAGegen{16-18F}{11-14F}
	{}
\SprungGegen{Schwache Spr�nge}
	{Schr�der (�berruf = Zweif�rber mit anderer UF bzw. OF), Unusual 2\SA{} auf OF-Er�ffnung}
%
\GegenEinSA
	{DONT (X = Einf�rber)}
	{}
	{}
\AndereGegenreizungen
	{Negativkontra (zeigt nach Zwr. auf 2er-St. 10+FP)}
	{neue Farbe auf 2er-Stufe nach Zwr. ist nicht forcierend}
	{Kontra gegen k�nstliche Gebote ist Ausspielmarke}
%
\SequenzHoechste{}
%\SequenzZweite{}
\InnereSequenzHoechste{}
%\InnereSequenzZweite{}
\AusspielDritteFuenfte
%\AusspielVierte
%\AusspielZweiteVierte
\AusspielSonstiges{Top of nothing (TON)}
\AusspielSA{4.-h�chste/TON, 2. der inneren Sequenz (bei 9 und 10)}
	{}
%\PositivHoch
\PositivNiedrig
%\PositivSonstiges{}
%\GeradeHoch
\GeradeNiedrig
%\GeradeSonstiges{}
\Abwuerfe{Lavinthal}
\MarkierungenSA{}
	{}
	%{\hspace{87mm} Vereinbarungen im Innenteil $\rightarrow$}
%
\Datum{\footnotesize \today}
%
\endMinikarteNeu

%\begin{twocolumn}
%
%\section*{Vereinbarungen}
%
%\begin{itemize}
%\item Supportkontra, Negativkontras, 3. (Unter-)Farbe forcing
%\item Assfrage ist generell RKCB 30/41
%\item 2/1 ist nach Zwischenreizung nonforcing
%\end{itemize}
%
%\end{twocolumn}

\end{document}
%    \end{macrocode}
