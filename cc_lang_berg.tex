\documentclass{article}
\usepackage{miniDBV2,amsthm}
\usepackage[a4paper, landscape, margin=5mm]{geometry}
\usepackage{multicol,tabularx}
\renewcommand{\familydefault}{cmss}
\begin{document}

\newtheorem{convention}{}

\hspace{2.6mm}
\beginMinikarteNeu
	{\large Kurt Lang}{\large Christoph Berg}
%
\Grundsystem{\large Rieneck Standard mod. (5er Oberfarben, Better Minor)}
\EinSA{\large15-17}{\large15-17}
%\EinSAkleinesSingle
%\EinSATopSingle
%\EinSAFuenferOFregel
\EinSAFuenferOFselten
\Mindestlaengen{\large3}{\large3}{\large5}{\large5}
\EinTreffBed{3+ \treff}
\EinTreffAnt{Inverted Minors (auch nach X)}
	{Schwache Sprünge}
\EinKaroBed{3+ (meist 4+) \karo}
\EinKaroAnt{dito}
	{}
\EinCoeurBed{5+ \coeur}
\EinCoeurAnt{2\SA{} Jacoby (PF, guter Fit), danach 3er-Stufe Kürzen, sonst Cuebids}
	{3\treff/3\karo{} Bergen (danach Cuebids), 3\coeur/4\coeur{} sperrend}
\EinPikBed{5+ \pik}
\EinPikAnt{dito
	}
		{}
\EinSABed{nat., selten auch 14 gute}
\EinSAAnt{2\treff{} Stayman oder {\bf 8-10F einl.}, Transfers
	(2\pik{} = \treff, 2\SA{} = \karo), alles ab 0F}
	{3x Schlemmint., 4\treff{} Assfrage}
%
\ZweiTreffBed
	{Partieforcing, OF-Semiforcing, \SA 23+}
	{}
\ZweiTreffAnt{2\karo{} Relais, Rest nat; 2x: OF-SF, 3x: PF}
\ZweiKaroBed
	{Multi: OF-Weak Two, UF-Semiforcing, \SA 20-22}
	{}
\ZweiKaroAnt{2\coeur{} Relais, 2\SA fragt (Antw: Min\coeur/Min\pik/Max\pik/Max\coeur), Rest nat; 3UF: SF}
\ZweiCoeurBed
	{Zweifärber mit \coeur (ca. 4-10F)}
	{}
\ZweiCoeurAnt{2\SA{} fragt}
\ZweiPikBed{Zweifärber mit in \pik + UF (ca. 4-10F)}
	{}
\ZweiPikAnt{dito}
\ZweiSABed{Zweifärber in UF (ca. 4-10F)}
	{}
\ZweiSAAnt{}
%
\BesondereZweierUndHoeher{3\SA: Gambling (stehende 7er-UF ohne Nebenwerte)}
	{4\SA: 6-5+ in UF}
%
\InfoKontraAb{12}
\InfoKontraOF
%\InfoKontraWerte
\FarbGegenEiner{8}{16}
\FarbGegenZweier{10}{17}
\StilDerGegenreizung{kompetitiv}
\Weiterreizung{Farbwechsel nonforcing}
\EinSAGegen{poln. 8-15F}{11-14F}
	{Polnischer SA: 4er OF, längere UF}
\SprungGegen{Weak Jumps}
	%{Michaels (Überruf = \pik{} + weitere, 2\SA{} = untere)}
	{Michaels}
%
\GegenEinSA
	%{Gegen starke \SA: DONT (X: Einfärber), sonst:}
	{Multi-Landy: 2\treff: beide OF, 2\karo: OF-Einfärber, 2\OF: 5-4+ OF-UF, }
	{\hphantom{Multi-Landy:} 2\SA: beide UF, 3\UF: UF-Einfärber, X: Strafe}
	{}
\AndereGegenreizungen
	{gegen 2\karo-Multi: X: Info-Kontra gegen \coeur-Weak Two}
	%{Lebensohl nach Info-X; 2\SA als Zweifärber in 4. Hand und Good-Bad NT}
	%{gegen starke \treff: 1x nat, 2x/3x Multi-Landy}
	{gegen starke \treff: Crash}
	{}
%
\SequenzHoechste{}
%\SequenzZweite{}
\InnereSequenzHoechste{}
%\InnereSequenzZweite{}
\AusspielDritteFuenfte
%\AusspielVierte
%\AusspielZweiteVierte
\AusspielSonstiges{Double hoch}%, K $\Rightarrow$ Längenm.}
\AusspielSA{4.-höchste (vom 3er \emph{x{\bf X}x} bzw. \emph{Fx{\bf X}})}
	{}
%\PositivHoch
\PositivNiedrig
%\PositivSonstiges{}
%\GeradeHoch
\GeradeNiedrig
%\GeradeSonstiges{}
\Abwuerfe{Lavinthal}
\MarkierungenSA{}
	{\hspace{87mm} Vereinbarungen im Innenteil $\rightarrow$}
%
\Datum{\footnotesize \today}
%
\endMinikarteNeu

\begin{twocolumn}

\section*{Vereinbarungen}

\begin{itemize}
\item Alle Punktspannen, insbesondere bei Gegenreizungen und Sperreröffnungen,
 können regelmä\ss ig auch etwas schwächer sein
 \begin{itemize}
 \item Ein Weak Two hat üblicherweise mindestens 3F in der Farbe
 \end{itemize}
\item Supportkontra, Negativkontras, 3. (Unter-)Farbe forcing
\item Lebensohl
 \begin{itemize}
 \item Nach 1\SA-Eröffnung
 \item Nach Weak Two des Gegners und unserem Kontra (z.B. (2\coeur)-X-(p)-3\SA!)
 \item Nach einfacher Hebung des Gegners und unserem Kontra (z.B. (1\pik)-X-(2\pik)-2\SA!)
 \item Selten auch Good-Bad-2NT
 \end{itemize}
\item Moderateur nach Reverse: Farbhebung schwach, 2NT/Rest = stark
\item Assfrage ist generell RKCB 30/41, meist 4\SA
 \begin{itemize}
 \item Weiter mit Zahl der Könige (0/1/2/3)
 \item 4 UF ist Assfrage in UF wenn es keine sperrende/kompetitive Situation ist
  (1\treff-4\treff{} ist sperrend, 1\treff-2\treff-3\treff-4\treff{} ist Assfrage)
 \item Weiter mit Zahl der Könige (0/1/2/3)
 \item In Unterfarbe ist 4\SA{} Signoff
 \item DOPI-ROPI (X = 30, P = 41)
 \end{itemize}
\item 1\karo-1\pik-2\karo-3\pik{} ist einladend
\item Nach 2\SA-3\treff-3\karo{} zeigt 4\treff{} beide OF 4-4
\item 2/1 ist nach Zwischenreizung nonforcing
\item Nach 1OF und Zwischenreizung ist 2SA nat, Überruf ist mind. einladend
\end{itemize}

\end{twocolumn}

\end{document}
