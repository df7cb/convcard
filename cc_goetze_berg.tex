\documentclass{article}
\usepackage[utf8x]{inputenc}
\usepackage[T1]{fontenc}
\DeclareUnicodeCharacter{9829}{$\heartsuit$}
\DeclareUnicodeCharacter{9830}{$\diamondsuit$}
\usepackage{amsthm}
\usepackage[a4paper, landscape, margin=5mm]{geometry}
\usepackage{multicol,tabularx}
\renewcommand{\familydefault}{cmss}
\begin{document}
\begin{multicols}{3}

\begin{tabular}{|l|}
\hline Gegenreizung und kompetitive Reizung \\
\hline Überrufe (Stil, Antworten, Reopening) \\
\hline 1SA Überruf (2./4. Position, Antworten, Reopening) \\
\hline Sprunggegenreizung (Stil, Antworten, Unusual NT) \\
\hline Cue-Bid + Sprung Cue-Bid (Stil, Antworten, Reopening) \\
\hline Gegen 1 SA (stark, schwach, 2./4. Hand) \\
\hline Gegen Sperransagen (Kontras, Cue-Bids, Sprünge) \\
\hline Gegen starke Treff und andere künstliche Eröffnungen \\
\hline Nach Negativ-Kontra des Gegners \\
\hline \end{tabular}

\begin{tabular}{|l|l|l|}
\hline Ausspiele und Markierung \\
\hline Ausspiele (grundsätzlich) \\
\hline & Ausspiel & In Partners Farbe \\
\hline Farbe & \\
\hline SA & \\
\hline Nachfolg. & \\
\hline Andere: \\

\hline Ausspiele \\
\hline Ausspiel & Gegen Farbkontrakte & Gegen SA \\
\hline As \\
\hline König \\
\hline Dame \\
\hline Bube \\
\hline 10 \\
\hline 9 \\
\hline Hoch-x \\
\hline Klein-x \\

\hline Reihenfolge der Markierung \\
\hline Partners Aussp. & Gegners Aussp. & Abwurf \\
\hline Farbe 1 \\
\hline 2 \\
\hline 3 \\
\hline SA 1 \\
\hline 2 \\
\hline 3 \\
\hline Markierungen (inklusive Trumpffarbe): \\

\hline Kontras \\
\hline Informationskontra (Stil; Antworten; Reopening) \\
\hline Negativ-Kontra, Kompetitiv-Kontra und weitere (Re-) Kontras \\
\hline \end{tabular}

\begin{tabular}{|l|}
\hline Deutsche Konventionskarte \\
\hline ♠ ♥ DBV e.V. ♦ ♣ \\
\hline Kategorie: \\
Club: Turnier: \\
\hline Paar: \\

\hline SYSTEM Zusammenfassung \\
\hline Genereller Stil \\
\hline 1 SA Eröffnung: \\
\hline 2 über 1 Antworten: \\
\hline Gebote, die besondere Gegenreizungen erfordern \\
\hline Forcing Pass Sequenzen \\
\hline Wichtige sonstige Bemerkungen \\
\hline Bluffs \\
\hline \end{tabular}

\end{multicols}

\begin{tabular}{|c|c|c|c|l|l|l|l|}
\hline Eröffnung &
 X wenn künstlich &
 Min. Anz. Karten &
 Negativ-X bis &
 BESCHREIBUNG &
 ANTWORTEN &
 WEITERREIZUNG &
 ÄNDERUNGEN ALS GEPASSTE HAND \\
\hline 1♣ & & & & & & & \\
\hline 1♦ & & & & & & & \\
\hline 1♥ & & & & & & & \\
\hline 1♠ & & & & & & & \\
\hline 1 SA & & & & & & & \\
\hline 2♣ & & & & & & & \\
\hline 2♦ & & & & & & & \\
\hline 2♥ & & & & & & & \\
\hline 2♠ & & & & & & & \\
\hline 2 SA & & & & & & & \\
\hline 3♣ & & & & & & & \\
\hline 3♦ & & & & & & & \\
\hline 3♥ & & & & & & & \\
\hline 3♠ & & & & & & & \\
\hline 3 SA & & & & & & & \\
\hline 4♣ & & & & & & & \\
\hline 4♦ & & & & & & & \\
\hline 4♥ & & & & & & & \\
\hline 4♠ & & & & & & & \\
\hline Gebote auf hoher Stufe (inkl. Schlemmreizung) & & & & & & & \\
\end{tabular}

\end{document}
