%    \begin{macrocode}
\documentclass{article}
\usepackage{miniDBV,amsthm}
\usepackage[a4paper, landscape, margin=5mm]{geometry}
\usepackage{multicol,tabularx}
\renewcommand{\familydefault}{cmss}
\begin{document}

\newtheorem{convention}{}
\newcommand{\Ref}[1]{$^{\ref{#1}}$}

\hspace{2.6mm}
\beginMinikarteNeu
	%{\large Christoph Berg}{\large Frank Luithle}
	{}{}
%
\Grundsystem{\large 5er Oberfarben (Inverted, Stenberg, Bergen)}
\EinSA{\large15-17F}{\large15-17F}
%\EinSAkleinesSingle
%\EinSATopSingle
%\EinSAFuenferOFregel
%\EinSAFuenferOFselten
\Mindestlaengen{\large3}{\large3}{\large5}{\large5}
\EinTreffBed{min. 3er, 12+F}
\EinTreffAnt{2/3\treff: Inverted\Ref{inverted}, 2\OF: 6er 5-8F,
	3\karo-\pik: Fit Jump}
\EinKaroBed{min. 3er (4er au"ser bei 4432-Verteilung), 12+F}
\EinKaroAnt{2/3\karo: Inverted\Ref{inverted}, 2\OF: 6er 5-8F}
\EinCoeurBed{min. 5er, 12+F}
\EinCoeurAnt{2\pik: 6er 5-8F, 2\SA: Stenberg\Ref{stenberg}, 3\UF/\coeur/4\coeur: Bergen\Ref{bergen}
	}
\EinPikBed{min. 5er, 12+F}
\EinPikAnt{2\SA: Stenberg\Ref{stenberg}, 3\UF/\pik/4\pik: Bergen\Ref{bergen}
	}
\EinSABed{ausgeglichen, 15-17F}
\EinSAAnt{2\treff: Stayman (0+F), 2\karo-\pik: Transfer\Ref{transfer} (0+F),
	4\treff: Gerber (04/1/2/3)}
%
\ZweiTreffBed
	{Benjamin: bel. Semiforcing mit 6er-Farbe / 8 Spielstichen (16+F), \SA\ 22-23F, \SA\ 26-27F}
	{}
\ZweiTreffAnt{2\treff: Relais}
\ZweiKaroBed
	{Benjamin: bel. Partieforcing mit 6er-Farbe / 9 Spielstichen (18+F), \SA\ 24-25F, \SA\ 28+F}
	{}
\ZweiKaroAnt{2\coeur: Relais}
\ZweiCoeurBed
	{Weak Two in \coeur\ (6-10F)}
	{}
\ZweiCoeurAnt{2\SA: Ogust\Ref{ogust}}
\ZweiPikBed{Weak Two in \pik\ (6-10F)}
	{}
\ZweiPikAnt{2\SA: Ogust\Ref{ogust}}
\ZweiSABed{ausgeglichen, 20-21F, 5er \OF\ und bel. 5422-Verteilung m�glich}
	{}
\ZweiSAAnt{3\treff: Stayman (0+F), 3\karo/\coeur: Transfer f�r \OF (0+F),
	4\karo: 5-5+ in \OF\ (schwach)}
%
\BesondereZweierUndHoeher{3\SA: Gambling\Ref{gambling} (stehendes 7er \UF ohne Nebenwerte)}
	{4\SA: 6-5+ in \UF}
%
\InfoKontraAb{12F}
\InfoKontraOF
%\InfoKontraWerte
\FarbGegenEiner{8}{15}
\FarbGegenZweier{10}{16}
\StilDerGegenreizung{in der Regel konstruktiv, schwache Hebungen}
\Weiterreizung{Farbw.\ nonf., �berr.\ fr.\ Stopper ($\rightarrow$\SA),
	zeigt Fit/stark Einf.}
\EinSAGegen{15-18 F}{11-14 F}{relativ ausgeglichen, Stopper, Weiterreizung wie nach 1 \SA-Er�ffnung}
\SprungGegen{Weak Jumps: schwach, sperrend}
	{Michaels Pr�zis:
	\UF\ $\rightarrow$ 2\karo: \OF, 2\SA: aUF+\coeur;
	\OF\ $\rightarrow$ �berruf: \treff+aOF, 2\SA: \UF, 3\treff: \karo+aOF}
%
\GegenEinSA{Multi-Landy: X: 5-4+ in UF-OF, 2\treff: 5-4+ \OF\ (2\karo: gleiche L�nge in \OF),
	2\karo: OF-Einf�rber,}
	{\hphantom{Multi-Landy:} 2\OF: 5-4+ \OF-UF, 2\SA: 5-5+ in \UF, 3\UF: UF-Einf�rber}
	{gegen schwachen \SA: X: mind. gleiche St�rke}
\AndereGegenreizungen
	{gegen 2\karo-Multi: X: Info-Kontra gegen \coeur-Weak Two}
	{Lebensohl nach Info-X; 2\SA\ als Zweif�rber in 4. Hand}
	{}
%
\SequenzHoechste{}
%\SequenzZweite{}
%\InnereSequenzHoechste{}
\InnereSequenzZweite{}
\AusspielDritteFuenfte
%\AusspielVierte
%\AusspielZweiteVierte
\AusspielSonstiges{Double hoch}%, K $\Rightarrow$ L�ngenm.}
\AusspielSA{4.-h�chste, 10/9 verspricht 0 oder 2 h�here}
	{}
%\PositivHoch
\PositivNiedrig
%\PositivSonstiges{}
%\GeradeHoch
\GeradeNiedrig
%\GeradeSonstiges{}
\Abwuerfe{Direkte Marke}
\MarkierungenSA{Lavinthal}
	{}
%
\Datum{\today}
%
\endMinikarteNeu

\newpage

\setlength{\parindent}{0pt}
\setlength{\parskip}{1ex}
%\setlength{\columnseprule}{.4pt}

\begin{multicols*}{3}

\section*{Konventionen zum System}

\begin{convention}[Inverted Minors] \label{inverted} ~

\begin{tabularx}{\columnwidth}{|lllX|}
\hline
1\UF & 3\UF & 6-9F & \\
     & 2\UF & 10-11+F & \\
2\SA &      & 12-14F & \\
3\UF &      & 12-14F & \\
sonst. &    & 14-15+F & zeigt Werte \\
\hline
\end{tabularx}
\end{convention}

\begin{convention}[Stenberg] \label{stenberg} ~

\begin{tabularx}{\columnwidth}{|lllX|}
\hline
1\OF & 2\SA & 12+F & 4+ \OF \\
3\OF &      & 16+F & keine K�rze \\
3\SA &      & 14-15F & keine K�rze \\
4\OF &      & 11-13F & keine K�rze \\
3\treff-3\pik &    & & K�rze \\
     & Relais & & Chicane-Frage + RKC \\
1. Stufe &  & & Chicane (Relais: RKC) \\
2+. Stufe &  & & Single, rollend RKC \\
\hline
\end{tabularx}
\end{convention}

\begin{convention}[Bergen] \label{bergen} ~

\begin{tabularx}{\columnwidth}{|lllX|}
\hline
1\OF & 3\treff & 9-11F & 4er \OF \\
     & 3\karo & 7-9F & 4er \OF \\
     & 3\OF & 0-6F & 4er \OF \\
     & 4\OF & 0-9F & 5er \OF \\
\hline
\end{tabularx}
\end{convention}

\begin{convention}[Jacoby-Transfer] \label{transfer} ~

\begin{tabularx}{\columnwidth}{|llllX|}
\hline
1\SA & 2\karo & 0+F & 5+ \coeur & \\
     & 2\coeur & 0+F & 5+ \pik & \\
     & 2\pik & 0+F & 5-5+ \UF\ oder 6er UF & \\
2\SA &       & & \karo-Pr�ferenz & \\
3\treff &    & & \treff-Pr�ferenz & \\
\hline
\end{tabularx}
\end{convention}

\begin{convention}[Ogust] \label{ogust}
Auch nach schw. Spr�ngen in 2er Stufe.

\begin{tabularx}{\columnwidth}{|lllX|}
\hline
2\karo-\pik & 2\SA &       & Frage nach Qualit�t \\
3\treff &      & 6-8F  & schlechte Farbe \\
3\karo  &      & 6-8F  & gute Farbe (2 Topfiguren) \\
3\coeur &      & 9-10F & schlechte Farbe \\
3\pik   &      & 9-10F & gute Farbe (2 Topfiguren) \\
3\SA    &      & 9-10F & AKD \\
\hline
\end{tabularx}

\end{convention}

\vfill
\columnbreak

\begin{convention}[Gambling] \label{gambling} ~

\begin{tabularx}{\columnwidth}{|lllX|}
\hline
3\SA    & 4\treff      & zum spielen oder ausbessern & \\
        & 4\OF         & zum spielen & \\
        & 5\UF/6\treff & zum spielen oder ausbessern & \\
        & 4\karo       & K�rzenfrage & \\
4\OF    & & \OF-K�rze & \\
4\SA    & & 7222-Verteilung & \\
5\treff & & \karo-K�rze & \\
5\karo  & & \treff-K�rze & \\
\hline
\end{tabularx}
\end{convention}

\begin{convention}[Schlemmkonventionen] \label{schlemm}
$\circ$ DOPI-ROPI \\
$\circ$ Mixed Cuebids \\
$\circ$ Roman Key Card 14/30, danach rollend nach Trumpfdame und platzierten
 K�nigen \\
$\circ$ Exclusion Key Card: ungew�hnlicher Sprung meist in 5er Stufe \\
\end{convention}

\end{multicols*}

\end{document}
%    \end{macrocode}
