%    \begin{macrocode}
\documentclass{article}
\usepackage{miniDBV2}
\usepackage[a4paper, landscape, margin=5mm]{geometry}
\usepackage{multicol,tabularx,amsthm}
\renewcommand{\familydefault}{cmss}

%\newtheorem{convention}{}
\newcounter{conv}
\setcounter{conv}{0}
\newenvironment{convention}[2]{{\bf \refstepcounter{conv}\label{#2}\theconv\ #1.}}{}
\newcommand{\Ref}[1]{$^{\ref{#1}}$}

\newcommand{\s}{$\leadsto$}

\begin{document}

\hspace{2.6mm}
\beginMinikarteNeu
	{\large Christoph Berg}{\large Frank Luithle}
	%{}{}
%
\Grundsystem{{\large 5er-Oberfarben}}
\EinSA{\large15-17 FP}{\large15-17 FP}
%\EinSAkleinesSingle
%\EinSATopSingle
\EinSAFuenferOFregel
%\EinSAFuenferOFselten
\Mindestlaengen{\large3}{\large3}{\large5}{\large5}
\EinTreffBed{min. 3er, 12\pl FP}
\EinTreffAnt{1\karo: Walsh\Ref{walsh}, 2\treff/2\SA/3\treff: Inverted, 2\karo: 5-5 \OF\ (2-5 FP),}
	{schwache Spr�nge}
\EinKaroBed{min. 3er (4er au"ser bei 4432-Verteilung), 12\pl FP}
\EinKaroAnt{2\karo/2\SA/3\karo: Inverted,}
	{schwache Spr�nge}
\EinCoeurBed{min. 5er, 12\pl FP}
\EinCoeurAnt{2\treff: 3er (10\pl FP), 2\karo: 5er (10\pl FP), 2\pik: 6er (5-8 FP), 2\SA: Partieforcing\Ref{majorgf},}
	{Bergen-Hebungen (3\treff: 9$^+$-11 FP), 3\pik: bel. Chicane, 3\SA: \pik-Splinter}
\EinPikBed{min. 5er, 12\pl FP}
\EinPikAnt{2\treff: 2er (10\pl FP), 2\karo/\coeur: 5er (10\pl FP), 2\SA: Partieforcing\Ref{majorgf},}
	{Bergen-Hebungen, 3\coeur: bel. Chicane, 3\SA: \coeur-Splinter}
\EinSABed{ausgeglichen, 15-17 FP, 5er \OF\ m�glich}
\EinSAAnt{Stayman (0\pl FP), 4-Farben-Transfer\Ref{transfer} (0\pl FP)}
	{4\treff: Gerber (04/1/2/3), 4\karo: 5-5\pl \OF}
%
\ZweiTreffBed{\karo-Weak Two,
	bel. Semiforcing, \SA\ 22-23/26-27 FP}
	{}
\ZweiTreffAnt{2\karo: Relais, 2\SA: Ogust, Rest: nat, forcierend}
\ZweiKaroBed{\coeur-Weak Two,
	bel. Partieforcing, \SA 24-25/28\pl FP}
	{}
\ZweiKaroAnt{2\coeur: Relais, 2\SA: Ogust}
\ZweiCoeurBed
	{Zweif�rber mit \coeur\ (6-10 FP, 5-5\pl \coeur/?, in 3. Hand auch 5-4)}
	{}
\ZweiCoeurAnt{2\pik: Spielen oder Ausbessern, 2\SA: Frage nach 2. Farbe/St�rke\Ref{zweifaerber}}
\ZweiPikBed{\pik-Weak Two}
	{}
\ZweiPikAnt{2\SA: Ogust}
\ZweiSABed{ausgeglichen, 20-21F, 5er \OF\ m�glich}
	{}
\ZweiSAAnt{3\treff: Puppet-Stayman\Ref{puppet} (4\pl FP), 3\karo/\coeur: Transf. \ra OF (4\pl FP),
	4\karo: 5-5\pl in \OF\ (schwach)}
%
\BesondereZweierUndHoeher{3\UF: 7\pl \UF (5-10 FP, in NG/3. Hand 6er m�glich),
	3\SA: Gambling\Ref{gambling} (in 3./4. Hand zum Spielen)}
	{4\UF: stehendes 7er-\OF\ mit Nebenwert, 4\SA: 6-5\pl in \UF}
%
%
% Gegenreizung und Verteidigung
% 5er-OF (cc_walsh.tex) und Moscito (cc_moscito05.tex)
%
\InfoKontraAb{12 FP}
\InfoKontraOF
%\InfoKontraWerte
\FarbGegenEiner{8}{15}
\FarbGegenZweier{10}{16}
\StilDerGegenreizung{in der Regel konstruktiv, schwache Hebungen}
\Weiterreizung{Farbw.\ nonf., �berr.\ fr.\ Stopper ($\rightarrow$\SA),
	zeigt Fit/stark Einf.}
\EinSAGegen
	%{15-18 FP}
	{k"unstl. (s.u.)}
	{11-14 FP}
	{relativ ausgeglichen, Stopper, Weiterreizung wie nach 1 \SA-Er�ffnung}
\SprungGegen{schwache Sprung"uberrufe\se Michaels Pr"azis: (1UF)-2\karo = beide OF}
	{(1OF)-2OF = \treff $+$ aOF\se (1OF)-3\treff = \karo $+$ aOF\se 2\SA = Unusual Notrump (niedrige Restf.)}
%
\GegenEinSA{Multi-Landy (Woolsey): X = 5-4 UF-OF\se 2\treff = beide OF 5-4\pl{}\se 2\karo = OF-Einf�rber}
	{\hphantom{Multi-Landy (Woolsey):} 2OF = diese OF + UF 5-4\pl{}\se 2\SA = 5-5\pl UF\se 3UF = nat. Sperre}
	{gegen 1\SA schwach (d.h. immer $<$16 FP): X = straforientiert, Rest wie oben}
\AndereGegenreizungen
	{gegen Multi \ra X direkt = Takeout gegen \coeur-Weak Two\se Sandwich Notrump}
	{1\SA polnisch = ca. 10-15 FP, Zweif"arber mit 4er-OF und 5\pl{}er-UF, verneint Verteilung f"ur Info-X}
	{gegen k"unstliche 1\treff: X = beide OF konstruktiv, 1$x$ = nat., 1\SA = beide UF, 2$x$ wie gegen starken SA}
%
\SequenzHoechste{}
%\SequenzZweite{}
%\InnereSequenzHoechste{}
\InnereSequenzZweite
\AusspielDritteFuenfte
%\AusspielVierte
%\AusspielZweiteVierte
\AusspielSonstiges{vom Double hoch}
\AusspielSA{4.-h�chste, 10/9 verspricht 0 oder 2 h�here}
	{}
%\PositivHoch
\PositivNiedrig
%\PositivSonstiges{}
%\GeradeHoch
\GeradeNiedrig
%\GeradeSonstiges{}
\Abwuerfe{Abw"urfe sind direkte Marken (niedrig = Interesse), Figur zeigt Karte darunter}
\MarkierungenSA{Lavinthal-Abw"urfe}
	{}
%

%
\Datum{\today}
%
\endMinikarteNeu

\newpage

\setlength{\parindent}{0pt}
\setlength{\parskip}{1ex}
%\setlength{\columnseprule}{.4pt}

\begin{multicols*}{3}

\section*{Konventionen zum System}

\begin{convention}{Walsh}{walsh} ~

\begin{tabularx}{\columnwidth}{|lllX|}
\hline
1\treff & 1\karo & 6-7 & (1) ausgegl. ohne 4er-\OF \\
        &        &      & \hspace{1em} (3er-\karo\ m�glich) \\
     &      & 6\pl & (2) \karo-Einf�rber \\
     &      & 12\pl & (3) 5-4\pl in \karo/OF \\
1\OF &      &      & 5-4\pl \treff/OF \\
1\SA &      & 12-14 & ausgegl., 4er-\OF\ m�glich \\
     & 2\treff & & UF-Schlemmint., Vertfr. (2) \\
     & 2\karo & 6-9 & zum Spielen (1/2) \\
     & 2\OF & & Partieforcing (3) \\
     & 2\SA & 8-9 & nat�rlich (2) \\
     & 3\treff/\OF &  & \karo-Autosplinter (2) \\
     & 3\karo &  & einladend (2) \\
\hline
\end{tabularx}

Weiterreizung nach Verteilungsfrage:

\begin{tabularx}{\columnwidth}{|llX|}
\hline
%1\treff & 1\karo & \\
%1\SA & 2\treff & \\
2\karo & & 3er-\karo, nicht 4333-Vert. \\
       & 2\coeur & Frage nach \treff-L�nge (2\pik: 5er-\treff, 2\SA: 4er-\treff) \\
2\coeur & & 3424-Vert. \\
2\pik & & 4324-Vert. \\
2\SA & & bel. 4333-Vert. \\
     & 3\treff & Frage nach 4er (3\karo: 4er-\treff, 3\OF: 4er-\OF) \\
3\treff & & 3325-Vert. (3\karo: KCB) \\
3\karo & & 4423-Vert. \\
\hline
\end{tabularx}
\end{convention}

\begin{convention}{Relais-Transfer}{relaistrf}
Nach 1\SA-R�ckgebot ist 2\treff\ Transfer \ra 2\karo\ und
2\SA\ Transfer \ra 3\treff.

\begin{minipage}{\columnwidth}
\begin{tabularx}{\columnwidth}{|c|c|l|l|X|}
\hline
\multicolumn{2}{|c|}{\textbf{Haltung}} & \multicolumn{3}{c|}{\textbf{St"arke}}\\
\hline
\emph{OF} & \emph{NF} & schwach & einladend & stark \\
\hline
\hline
4OF  & ausg.  & pass  & 2\treff{}\s 2\SA & 3\SA\\
\hline
4OF  & 5\treff & 2\SA{}\s p &
                \emph{nicht zeigb.} &
                nach 1\treff: 2\SA{}\s 3\karo{}\\
4OF  & 5\karo & 2\treff{}\s p &
                2\karo{}\footnote{genau einladend, sonst Walsh} &
                nach 1\karo: 2\SA{}\s 3\karo{}\\
\hline
5OF  & --    & 2OF  &
                2\treff{}\s 2OF{}\footnote{mindestens einladend} &
                2\treff{}\s 3\SA\\
5OF  & 5$x$ & 2OF & 2\treff{}\s 3$x$ & 3$x$\\
5\pik & 4\coeur & 2\coeur & 2\treff{}\s 2\coeur & 2\SA{}\s 3\coeur\\
\hline
6OF  & --    & 2OF  & 2\treff{}\s 3OF & 3OF{}\footnote{leichtes Schlemm-Interesse}\\
      & Sngl  &       &             & 4$x$\\
      & Chic  &       &             & 2\treff{}\s 4$x$\\
\hline
\end{tabularx}
\centerline{\emph{OF = Oberfarbe des Antwortenden, NF = Nebenfarbe}}%
\end{minipage}

\end{convention}

%\begin{convention}{Inverted Minors}{inverted} ~
%
%\begin{tabularx}{\columnwidth}{|lllX|}
%\hline
%1\UF & 2\SA & 2-6F & \\
%     & 3\UF & 7-9F & \\
%     & 2\UF & 10-11+F & \\
%2\SA &      & 12-14F & \\
%3\UF &      & 12-14F & \\
%sonst. &    & 14-15+F & zeigt Werte \\
%\hline
%\end{tabularx}
%\end{convention}

\vfill
\columnbreak

\begin{convention}{Partieforcing}{majorgf}
Weiterreizung nach 1\OF-2\SA:

\begin{tabularx}{\columnwidth}{|lllX|}
\hline
1OF & 2\SA & 12\pl & 4\pl OF \\
\hline
3OF     &         & 14\plus-16 & keine K�rze \\
3aOF    &         & 14\plus-16 & bel. Chicane (Rel fragt) \\
3\SA    &         & 14\plus-16 & Single in aOF \\
4\UF    &         & 14\plus-16 & Single \UF \\
\hline
3\treff & 3\karo  & 11-14\minus{} & K�rzenfrage, weiter wie oben \\
\hline
3\karo  & 3\coeur & 17\pl & K�rzenfrage, analog oben \\
3\pik   &         & 17\pl & bel. Chicane (Rel fragt) \\
4\coeur &         & 17-18 & keine K�rze \\
4\pik   &         & 19\pl & keine K�rze \\
\hline
\end{tabularx}
Danach weiter mit Cuebids, 4\SA ist KCB.
\end{convention}

%\begin{convention}{Bergen}{bergen} ~
%
%\begin{tabularx}{\columnwidth}{|lllX|}
%\hline
%1\OF & 3\treff & 9$^+$-11 & 4er-\OF \\
%     & 3\karo & 7-9$^-$ & 4er-\OF \\
%     & 3\OF & 0-6 & 4er-\OF \\
%     & 4\OF & 0-9 & 5er-\OF \\
%\hline
%\end{tabularx}
%\end{convention}

\begin{convention}{Jacoby-Transfer}{transfer}
Nach Transfer auf OF reizt der Er�ffner mit 4er-Anschluss und Maximum Double
(2\SA: Transferfarbe/4333), mit 4er-Anschl. und Minimum 3 in Farbe.

%Nach Transfer auf UF zeigt  der Er�ffner mit 2\SA \karo-Pr�ferenz, mit 3\treff
%\treff-Pr�ferenz.
\end{convention}

%\begin{convention}{Ogust}{ogust}
%Auch nach schw. Spr�ngen in 2er-Stufe.
%N�chste freie Farbe ist erst K�rzenfrage, danach PKCA\Ref{pkca}.
%
%\begin{tabularx}{\columnwidth}{|lllX|}
%\hline
%2\karo-\pik & 2\SA &       & Frage nach Qualit�t \\
%3\treff &      & 6-8  & schlechte Farbe \\
%3\karo  &      & 6-8  & gute Farbe (2 Topfiguren) \\
%3\coeur &      & 9-10 & schlechte Farbe \\
%3\pik   &      & 9-10 & gute Farbe (2 Topfiguren) \\
%3\SA    &      & 9-10 & AKD \\
%\hline
%\end{tabularx}
%
%\end{convention}

\begin{convention}{Zweif�rber}{zweifaerber}
Danach ist n�chste freie Farbe K�rzenfrage. Die 1./2. freie Farbe ist dann
PKCA\Ref{pkca} f�r die niedrige/hohe Farbe.

\begin{tabularx}{\columnwidth}{|llllX|}
\hline
2\coeur & 2\SA &       & Frage 2. Farbe/St�rke & \\
3\treff &      & 6-10 & 5-5 \coeur/\treff; 3\karo: weitere Frage & \\
	& 3\coeur & 6-8 & \karo-K�rze & \\
	& 3\pik   & 6-8 & \pik-K�rze & \\
	& 3\SA    & 8-10 & \karo-K�rze & \\
	& 4\treff & 8-10 & \pik-K�rze & \\
3\karo  &      & 6-8  & 5-5 \coeur/\karo & \\
3\coeur &      & 6-8  & 5-5 \coeur/\pik\ (4\treff/\karo: \coeur/\pik-Basis) & \\
3\pik   &      & 9-10 & 5-5 \coeur/\pik\ (dto.) & \\
3\SA    &      & 9-10 & 5-5 \coeur/\karo & \\
\hline
\end{tabularx}
\end{convention}

\vfill
\columnbreak

\begin{convention}{Preempt Keycard Ask (PKCA)}{pkca}

Nach allen schwachen Er�ffungen und Spr�ngen, meist mit 4\treff\ gestellt.
Rollend:

\begin{tabularx}{\columnwidth}{|lX|}
\hline
1. Stufe & 1 Ass ohne Trumpfdame \\
2. Stufe & 0 Asse \\
3. Stufe & 1 Ass mit Trumpfdame \\
4. Stufe & 2 Asse ohne Trumpfdame \\
5. Stufe & 2 Asse mit Trumpfdame \\
\hline
\end{tabularx}
\end{convention}

\begin{convention}{Puppet-Stayman}{puppet} ~

\begin{tabularx}{\columnwidth}{|llX|}
\hline
2\SA & 3\treff & Frage nach 5er- und 4er-\OF \\
3\karo  &      & 4er-\OF, kein 5er-\OF \\
	& 3\coeur & 4er-\pik\ (4er-\coeur\ m�glich) \\
	& 3\pik & 4er-\coeur \\
3\coeur &      & 5er-\coeur \\
3\pik   &      & 5er-\pik \\
3\SA    &      & kein 5er/4er-\OF \\
\hline
\end{tabularx}

Transfer-Annahme zeigt ein Double, 3\SA\ einen 3er-Anschluss.
Andere Gebote zeigen gute 3/4er-Anschl�sse und sind Cuebids.
\end{convention}

\begin{convention}{Gambling}{gambling} ~

\begin{tabularx}{\columnwidth}{|lllX|}
\hline
3\SA    & 4\treff      & zum Spielen oder Ausbessern & \\
        & 4\OF         & zum Spielen & \\
        & 5\treff{}/\karo{}/6\treff & zum Spielen oder Ausbessern & \\
        & 4\karo       & K�rzenfrage & \\
4\OF    & & \OF-K�rze & \\
4\SA    & & 7222-Verteilung & \\
5\treff & & \karo-K�rze & \\
5\karo  & & \treff-K�rze & \\
\hline
\end{tabularx}
\end{convention}

\begin{convention}{Schlemmkonventionen}{schlemm}
$\circ$ DOPI-ROPI \\
$\circ$ Mixed Cuebids, erstes 5er-Gebot zeigt Erstrundenkontrolle \\
$\circ$ Roman KCB 14/30, rollend nach Trumpf-D und platzierten K \\
$\circ$ Exclusion Keycard: ungew�hnlicher Sprung meist in 5er-Stufe \\
$\circ$ Josephine: 5\SA\ im Sprung fragt Topfiguren (0/1/2/3)
\end{convention}

\begin{convention}{Sonstiges}{sonstiges} ~

\begin{tabularx}{\columnwidth}{|lllX|}
\hline
1\karo  & 2\treff & & \\
2\karo  &         & 12-13 & ausgegl. oder nat�rlich \\
        & \ra 2\coeur &       & Frage \\
2\SA    &         & 18-19 & ausgegl. \\
\hline
\end{tabularx}
\end{convention}

% TODO:
% Stayman/Smolen
% 1karo-2treff
% 4way-Transfer

\end{multicols*}

\end{document}
%    \end{macrocode}
