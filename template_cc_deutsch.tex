\documentclass{article}
\usepackage[utf8x]{inputenc}
\usepackage[T1]{fontenc}
\DeclareUnicodeCharacter{9829}{$\heartsuit$}
\DeclareUnicodeCharacter{9830}{$\diamondsuit$}
\usepackage{amsthm}
\usepackage[a4paper, landscape, margin=5mm]{geometry}
\usepackage{multicol,tabularx,rotating}
\renewcommand{\familydefault}{cmss}
\begin{document}
\begin{multicols}{3}

\begin{tabular}{|l|}
\hline \multicolumn{1}{c}{\bf \large Gegenreizung und kompetitive Reizung} \\
\hline Überrufe (Stil, Antworten, Reopening) \\
\\
\\
\\
\\
\\
\\
\hline 1SA Überruf (2./4. Position, Antworten, Reopening) \\
\\
\\
\\
\\
\\
\hline Sprunggegenreizung (Stil, Antworten, Unusual NT) \\
\\
\\
\\
\\
\hline Cue-Bid + Sprung Cue-Bid (Stil, Antworten, Reopening) \\
\\
\\
\\
\\
\hline Gegen 1 SA (stark, schwach, 2./4. Hand) \\
\\
\\
\\
\hline Gegen Sperransagen (Kontras, Cue-Bids, Sprünge) \\
\\
\\
\\
\\
\\
\\
\\
\hline Gegen starke Treff und andere künstliche Eröffnungen \\
\\
\\
\\
\\
\hline Nach Negativ-Kontra des Gegners \\
\\
\\
\\
\\
\hline \end{tabular}

\begin{tabular}{|l|l|l|}
\hline \multicolumn{3}{c}{\bf \large Ausspiele und Markierung} \\

\hline \multicolumn{3}{c}{\bf Ausspiele (grundsätzlich)} \\
\hline & Ausspiel & In Partners Farbe \\
\hline Farbe & & \\
\hline SA & & \\
\hline Nachfolg. & & \\
\hline \multicolumn{3}{|l|}{Andere:} \\
       \multicolumn{3}{|l|}{} \\

\hline \multicolumn{3}{c}{\bf Ausspiele} \\
\hline Ausspiel & Gegen Farbkontr. & Gegen SA \\
\hline As & & \\
\hline König & & \\
\hline Dame & & \\
\hline Bube & & \\
\hline 10 & & \\
\hline 9 & & \\
\hline Hoch-x & & \\
\hline Klein-x & & \\

\hline \multicolumn{3}{c}{\bf Reihenfolge der Markierung} \\
\hline Partner & Gegner & Abwurf \\
\hline Farbe \hfill 1 & & \\
\hline       \hfill 2 & & \\
\hline       \hfill 3 & & \\
\hline SA    \hfill 1 & & \\
\hline       \hfill 2 & & \\
\hline       \hfill 3 & & \\
\hline \multicolumn{3}{|l|}{Markierungen (inklusive Trumpffarbe):} \\
       \multicolumn{3}{|l|}{} \\
       \multicolumn{3}{|l|}{} \\

\hline \multicolumn{3}{c}{\bf \large Kontras} \\
\hline \multicolumn{3}{|l|}{Informationskontra (Stil; Antworten; Reopening)} \\
       \multicolumn{3}{|l|}{} \\
       \multicolumn{3}{|l|}{} \\
       \multicolumn{3}{|l|}{} \\
       \multicolumn{3}{|l|}{} \\
\hline \multicolumn{3}{|l|}{Negativ-Kontra, Kompetitiv-Kontra und weitere} \\
       \multicolumn{3}{|l|}{(Re-)Kontras} \\
       \multicolumn{3}{|l|}{} \\
       \multicolumn{3}{|l|}{} \\
       \multicolumn{3}{|l|}{} \\
       \multicolumn{3}{|l|}{} \\
       \multicolumn{3}{|l|}{} \\
       \multicolumn{3}{|l|}{} \\
       \multicolumn{3}{|l|}{} \\
       \multicolumn{3}{|l|}{} \\
\hline \end{tabular}

\begin{tabular}{|l|}
\hline \multicolumn{1}{c}{\bf \Large Deutsche Konventionskarte} \\
\hline \multicolumn{1}{c}{\bf \Large ♠ ♥ DBV e.V. ♦ ♣} \\
\hline Kategorie: \\
\hline Club: \\
\hline Turnier: \\
\hline Paar: \\
\\

\hline \multicolumn{1}{c}{\bf \Large System-Zusammenfassung} \\
\hline Genereller Stil \\
\\
\\
\\
\\
\hline 1 SA Eröffnung: \\
\\
\hline 2 über 1 Antworten: \\
\\
\hline Gebote, die besondere Gegenreizungen erfordern \\
\\
\\
\\
\\
\\
\\
\\
\\
\\
\\
\\
\\
\\
\\
\\
\\
\\
\hline Forcing Pass Sequenzen \\
\\
\\
\\
\hline Wichtige sonstige Bemerkungen \\
\\
\\
\\
\hline Bluffs \\
\\
\hline \end{tabular}

\end{multicols}

\begin{tabular}{|c|c|c|c|l|l|l|l|}
\hline
 \begin{sideways}\bf Eröffnung\end{sideways} &
 \begin{sideways}X\,wenn\,künstlich\end{sideways} &
 \begin{sideways}Min.\,Anz.\,Karten\end{sideways} &
 \begin{sideways}Negativ-X\,bis\end{sideways} &
 \bf Beschreibung &
 \bf Antworten &
 \bf Weiterreizung &
 \bf Änderungen als gepasste Hand \\
\hline 1♣   & & & & & & & \\
\hline 1♦   & & & & & & & \\
\hline 1♥   & & & & & & & \\
\hline 1♠   & & & & & & & \\
\hline 1 SA & & & & & & & \\
\hline 2♣   & & & & & & & \\
\hline 2♦   & & & & & & & \\
\hline 2♥   & & & & & & & \\
\hline 2♠   & & & & & & & \\
\hline 2 SA & & & & & & & \\
\hline 3♣   & & & & & & & \\
\hline 3♦   & & & & & & & \\
\hline 3♥   & & & & & & & \\
\hline 3♠   & & & & & & & \\
\hline 3 SA & & & & & & \multicolumn{2}{l|}{\bf Gebote auf hoher Stufe (inkl. Schlemmreizung)} \\
\hline 4♣   & & & & & & \multicolumn{2}{l|}{} \\
\hline 4♦   & & & & & & \multicolumn{2}{l|}{} \\
\hline 4♥   & & & & & & \multicolumn{2}{l|}{} \\
\hline 4♠   & & & & & & \multicolumn{2}{l|}{} \\
\hline \end{tabular}

\end{document}
